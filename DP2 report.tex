\documentclass[letterpaper]{article}

\usepackage[margin=1.25in]{geometry}
\usepackage{graphicx}
\usepackage{clrscode4e}
\usepackage{listings}

\begin{document}

\section{Introduction}

When deployed to a disaster-stricken area, first responders need to be able to efficiently communicate their
location and other information to their base camp. However, communication infrastructure is often unreliable
or inaccessible during these crises. To solve this problem, we design an ad-hoc wireless network of portable
devices that allows responders to send messages to their base camp in a reliable, efficient, and safe way.
Our rigorous design accounts for the rapid changes in network topology that  occur as the responders move 
around, as well as the possible presence of malicious agents in the field. 

\subsection{Design Overview}

\subsection{Tradeoffs and Design Decisions}

\section{Design}

\subsection{Routing tables}

Initially, and every thirty seconds onwards, the network undergoes the following update procedure that 
discovers changes in the network topology:

\begin{itemize}
  \item Each node issues one scan() message to populate its neighbor table.
  \item The base sends a route update broadcast() message to its neighbors, notifying them they are one hop
        away from the base. Each of these nodes updates their routing tables.
  \item Each of the neighbors from step 2 repeats that step themselves, aggregating the cost metric
        appropriately. Each node discards every broadcast() it receives after the first. 
        This procedure continues until all nodes have received a broadcast, and takes time proportional 
        to the diameter of the network.
\end{itemize}

\noindent The addition of new nodes to the network is also handled by this procedure. After this procedure, 
each node has multiple paths to the base and can begin using the metrics to estimate which paths are best.


\subsection{Cost algorithm}

\subsection{Encryption}

\section{Analysis}

\section{Conclusion}

\section{References}

\end{document}
