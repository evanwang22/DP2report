\documentclass[letterpaper]{article}

\usepackage[margin=1.25in]{geometry}
\usepackage{graphicx}
\usepackage{clrscode4e}
\usepackage{listings}

\begin{document}

\section{Introduction}

When deployed to a disaster-stricken area, first responders need to be able to efficiently communicate their
location and other information to their base camp. However, communication infrastructure is often unreliable
or inaccessible during these crises. To solve this problem, we design an ad-hoc wireless network of portable
devices that allows responders to send messages to their base camp in a reliable, efficient, and safe way.
Our rigorous design accounts for the rapid changes in network topology that occur as the responders move 
around, as well as the possible presence of malicious agents in the field. 

\subsection{Design Overview}

\subsection{Tradeoffs and Design Decisions}

\section{Design}

\subsection{Routing tables}

An effective and efficient routing protocol is necessary for nodes to determine the best paths to send
packets to the base without constantly congesting the network.
\\

\noindent Our routing protocol is based on a link-state advertisement scheme in which neighboring nodes 
inform each other of incremental changes. Upon receiving an advertisement, a node will use the 
\textsc{broadcast} function to forward it to all its neighbors. As a result of this flooding process,
each node's routing table will contain a complete map of the network. Then, each node will independently
run a computation based on our cost metric to find the shortest routes to the base. As long as the nodes
have a consistent view of the topology and the same metric, resulting routes at different nodes will
correspond to a valid path.
\\

\noindent There are two data structures which our routing protocol uses: link-state advertisements and
routing tables. 
\\

\noindent Link-state advertisements have the following format:
\\

\noindent Routing tables
\\

\noindent Initially, the network undergoes the following procedure that discovers the network topology:

\begin{enumerate}
  \item Each node constructs its link-state advertisement by calling \textsc{scan}. 
  \item Nodes begin to flood the network with their LSAs, and build up their routing tables based on
  the advertisements they receive.
  \item Once all the LSAs have been discarded according to the routing table protocol, each node will have
  a complete map of the network. This will take time proportional to the diameter of the network, because
  LSAs must be propogated throughout the entire network.
\end{enumerate}

\noindent The addition of new nodes to the network is also handled by the above procedure. However, the
network utilization when adding one new node to the network is much smaller than the utilization of the 
initial procedure, because only one link-state advertisement created. Forwarding it  

\subsection{Cost algorithm}

\subsection{Encryption}

Jamming is a always a potential problem

\section{Analysis}

\section{Conclusion}

\section{References}

\end{document}
